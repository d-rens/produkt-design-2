\documentclass{report}

\setlength{\parindent}{0pt}
\usepackage{booktabs}
\usepackage[a4paper,margin=3cm]{geometry}

\title{Projekt Produktdesign Teil 2}
\author{Daniel Renschler}
\date{21. April $\to$ 14. Juni 2023}

\begin{document}
\maketitle

\tableofcontents
\clearpage


\section{Aufgabe}
\section{Beschreibung}
Sie hatten in Teil 1 des Projektes den Auftrag, einen Parf\"umflacon zu
entwerfen. Die technische Darstellung, Fertigung und entsprechende Aspekte wie
Werkstoffe und Verfahren blieben unber\"ucksichtigt.

Im zweiten Teil werden diese Aspekte nun genauer betrachtet und entsprechende
Entscheidungen getroffen vom Modellbau bis hin zur Umsetzung in CAD.


\subsection{Auftragsannahme}
Definieren Sie das Projekt, formulieren Sie ein Ziel und erstellen Sie eine
Planung. Diese beinhaltet eine Einleitung des Projekts in Arbeitspakete, zu
diesen gehoert auch jeweils die Erarbeitung der noetigen Theorie oder die
Beschaffung der noetigen Informationen, ausserdem auch die Dokumentation.

\textit{Ggf. muss das Ergebnis aus Teil 1 stark geaendert oder gar verworfen werden.}


\subsection{Dokumentation}
Die Dokumentation wird in Affinity Publisher erstellt, sier enthaelt ein
Deckblatt, ein Inhaltsverzeichnis, Seitennummerierung und einen Anhang.
\textbf{Beachten Sie die Vorgaben bei gleichwertigen Festellungen von
Schuelerleistungen!}

Die Dokumentation wird auf DIN A4 (hoch) doppelseitig angelegt und wird
gebunden. Die weitere Gestaltung der Dokumentation wird bewertet! Dazu gehoeren
insbesondere Layout, Farbwahl, Mikro und Makrotypograpfie einschliesslich Satz.
Beschriftung aller Zeichnungen.

Zur technischen Umsetzung der Dokumentation bestehen folgende Vorgaben:
\begin{itemize}
  \item Das Inhaltsverzeichnis darf manuell erstellt werden, Seitenzahlen nicht.
  \item Master-Seiten mit Grundlinienraster ist zu verwenden.
  \item Farben werden ueber Farbfelder eingestellt und vergeben.
  \item Direkte Formatierung ist zu vermeiden. (\"Uberall Absatz- und Zeichenstile verwenden!)
  \item Fotos werden mit mindestens 300 dpi platziert und m\"ussen auch
    ausserhalb des Dokuments klar auffindbar und bennant sein. (Am besten
    Verkn\"upfen und Verpacken statt einbetten!)
  \item Fotos d\"urfen nicht (unabsichtlich) gestaucht und gestreckt werden.
\end{itemize}

\subsection{Fertigung und Modellbau}
Modellieren Sie zunaechst Ihr Ergebnis aus Teil 1.
\begin{enumerate}
  \item[(a)] Machen Sie sich mit den zur Verf\"ugung stehenden Werkstoffen und Werkzeugen vertraut.
  \item[(b)] Analysieren Sie anhand Ihres Modells Teil 1 formale Zusammenh\"ange und Ergonomie.
  \item[(c)] Optimieren Sie Ihr Modell Teil 1 - schriftlich und mit Scribbles.
  \item[(d)] Modellieren Sie einen "Modell" Teil 2.
  \item[(e)] Fotografieren Sie den Prozess regelmaessig.
\end{enumerate}

\subsection{Optimierung und technische Darstellung}
\begin{enumerate}
  \item[(f)] Setzen Sie das "Modell" Teil 2 in CAD um.
    \begin{itemize}
      \item Verwenden Sie Teile, die Sie zu einer Baugruppe zusamensetzen (z.B.
        Korpus und Spruehkopf des Flancons)
      \item Weisen Sie auch Farben und Werkstoffe zu.
      \item Speichern Sie jedes Teil sowohl als SolidWorks-Teil als auch die
        finale Version als STEP, IGES und STL.
      \item Erstellen Sie von den Teilen und der Baugruppe jeweils Renderings in Axonometrie.
    \end{itemize}

  \item[(g)] Erstellen Sie eine technische Zeichnung mit Bemassung in sinnvoller
    Normalprojektion (DIN 5456 Projektionsmethode 1). Speichern Sie als
    SolidWorks-Zeichnung und PDF.
\end{enumerate}

\textit{Nur wer gewissenhaft und sauber arbeitet, kann sein Ergebnis am Ende
als Prototyp im 3D-Drucker ausdrucken.}

\section{Abgabe}
\subsection{Dokumentation:}
\begin{itemize}
  \item Projektdefinition, Projektplanung, min. ein dokumentiertes Vriefind mit Frau Hildebrand.
  \item Illustrierte Beschreibung der fuer den Modellbau zur Verfuegung stehenden Werkstoffe und -zeuge.
  \item Illustrierte Beschreibung des Modellbauprozesses.
    \vspace{20pt}
  \item Illustrierte Analyse des Modells Teil 1.
  \item Begrundung der Optimierung mit Scribbles.
    \vspace{20pt}
  \item Illustrierte Uebersicht ueber die CAD-Umsetzung mit Beschriftungen und kurzer Erlaeuterung.
  \item Begruendete Werkstoffwahl
  \item Beschreibung eines geeigneten Fertigungsverfahren
  \item Illustrierte Uebersicht ueber die CAD-Umsetzung mit Beschriftungen und kurzer Erlaeuterung.
    \vspace{20pt}
  \item Reflexion des Projekts mit Bezug auf Projektdefinition und Projektplanung.
\end{itemize}

\subsection{Modell}
\begin{itemize}
  \item Modell Teil 1
  \item Modell Teil 2
\end{itemize}

\subsection{Computer}
Dateien (ordentlich benannt und sortiert)
\begin{itemize}
  \item Dokumentation als PDF und verpacktes Affinity-Publisher-Projekt;
  \item Teile und Baugruppen als SolidWorks-Dateien, SETP, IGES und STL;
  \item Zeichnungen aller Teile und Baugruppen als SolidWorks-Zeichnung und PDF;
  \item Renderings aller Teile und Baugruppen als PNG, JPG, oder PDF (min. 1200$\cdot$1200px);
  \item Digitalisierungen der Scribbles als PNG, JPG oder PDF (min. 1200$\cdot$1200px);
  \item Fotos vom Modellbauprozess und der Modellanalyse als JPG (min. 1200$\cdot$1200px);
\end{itemize}

\section{Bewertung}

\begin{table}[htbp]
  \centering
  \begin{tabular}{@{}p{8cm}r@{}}
    \toprule
    \textbf{Kriterium} & \textbf{Punkte} \\
    \midrule
    Projektdefinition, Projektplanung und Reflexion & 10 \\
    \vspace{1pt} & \vspace{1pt} \\
    Optimierung und Dokumentation: \textit{Detaillierung, Aussagekraft,
  Eignung, Vollständigkeit, Vorgabeneinhaltung} & 10 \\
    \vspace{1pt} & \vspace{1pt} \\
    Werkstoffwahl und Fertigungsplanung: \textit{Detaillierung und
  Vollstaendigkeit bzw. Kriterien, Nachvollziehbarkeit, Eignung} & 8 \\
    \vspace{1pt} & \vspace{1pt} \\
    CAD-Umsetzung und Dokumentation: \textit{Vollstaendigkeit, Detaillierung,
  Eignung/technische Sauberkeit, Aussagekraft (Renderings)} & 18 \\
    \vspace{1pt} & \vspace{1pt} \\
     Modellbau und Dokumentation & 12 \\
    \vspace{1pt} & \vspace{1pt} \\
    Modellanalzse und Dokumentation & 12 \\
    \vspace{1pt} & \vspace{1pt} \\
    Dokumentation: \textit{Aufbau, Gestaltung, Sprache, Eignung, Vorgabeneinhaltung}& 10 \\
    \midrule
    Summe & 80 \\
    \bottomrule
  \end{tabular}
\end{table}





\end{document}
